% This is samplepaper.tex, a sample chapter demonstrating the
% LLNCS macro package for Springer Computer Science proceedings;
% Version 2.20 of 2017/10/04
%
\documentclass[runningheads]{llncs}
%
\usepackage{graphicx}
\graphicspath{ {./img/} }

\input{solidity-highlighting.tex}	% copy the file from this repo
% Used for displaying a sample figure. If possible, figure files should
% be included in EPS format.
%
% If you use the hyperref package, please uncomment the following line
% to display URLs in blue roman font according to Springer's eBook style:
% \renewcommand\UrlFont{\color{blue}\rmfamily}

\begin{document}
%
\title{Ants-Review: A Protocol For Open Anonymous Peer-Reviews}
%
%\titlerunning{Abbreviated paper title}
% If the paper title is too long for the running head, you can set
% an abbreviated paper title here
%
\author{Bianca Trovò\inst{1,2}\orcidID{0000-0002-6776-2304}\thanks{Correspondent author: bianca.trovo@alumni.unitn.it} \and
Nazzareno Massari\inst{3,4}\orcidID{0000-0002-6638-2174}}
%
\authorrunning{B. Trovò et N. Massari}
% First names are abbreviated in the running head.
% If there are more than two authors, 'et al.' is used.
%
\institute{Sorbonne Université, Faculté des Sciences et Ingénierie, 75005 Paris, France \and
Neurospin research center, CEA/SAC/DSV/I2BM, 91191 Gif-sur-Yvette, France
\email{bianca.trovo@alumni.unitn.it}\\
\and
Alma Mater Politecnico di Torino, 10129 Turin, Italy\and
Independent Blockchain Engineer, London, UK \\
\email{nazzareno@nazzarenomassari.com}}
%
\maketitle              % typeset the header of the contribution
%
\begin{abstract}
Peer-review is a necessary and essential quality control step for scientific publications. However, the process, which is very costly in terms of time investment, not only is not remunerated but it’s also not recognized by the academic community as a relevant scientific output for a researcher. Therefore, scientific dissemination is affected. Here, to solve this issue we propose a blockchain-based incentive protocol that rewards scientists also for their contributions to other scientists’ work and that builds up a reputational system. We designed a basic Bounty-like contract called AntsReview that allows any author to issue a call for peer-reviewing their scientific publication. If requirements are met, peer-reviews will be audited by an external editor and payed by the Issuer. To promote ethical behaviour the system will implement a quadratic funding on AntsReview.
\keywords{Blockchain  \and Open Science \and Peer-review \and Incentivization.}
\end{abstract}
%
%
\section{Introduction}
From its birth in 2008 as a distributed ledger for the peer-to-peer electronic cash system 'Bitcoin' \cite{Bitcoin}, blockchain technology has spread far beyond the sole cryptocurrency domain, in particular after the implementation of a general purpose smart contract functionality introduced by Ethereum. 
Besides a growing number of applications ranging from business, healthcare, music industry, government, identity to cite but a few, blockchain technology has recently started to catalyse the attention of the scientific community as well \cite{Bitcoin-Nature-focus,vanRossum2017-DigSci} with the promising potential of being a 'game changer' in outdated and broken scientific practices and leading towards Open Science \cite{AES}. In particular, a 'blockchainified science'\cite{BlockchainforScience} could 'reduce waste'\cite{ReducingWaste-Lancet}, by disclosing each step the research cycle to 'scientific self-correction' way before the final publication step and therefore help fixing the current reproducibility crisis in science. Scholars have pointed out how the intrinsic characteristics of blockchain technology (decentralization, for which there are no trusted third parties; cryptographic hashing, that creates a digital footprint able to keep a 'trail of research';  
to prevent vs publication bias (reporting only positive results) or p-hacking or ex-post-facto hypothesizing (changing the hypothesis after having looked into the data)




The paper is structured as follows. Sections 2 presents research background and related works. Section 3 explains the proposed Ethereum-based protocol illustrating the system architecture and the current status of implementation. In Sect. 4 we discuss the implication of our work. Section 5 contains the conclusions. Appendixes present the smart contracts. 

\section{Background}
\subsection{Peer-review: a flawed and outdated system}

"Peer review In recent years a lot of attention has been given to fundamental problems associated with the peer review process which lies at the heart of scholarly communication5,6,7.The perceived problems are multiple.There is a lack of visibility and recognition for reviewers, with their review work remaining largely unnoticed.Together, the increased number of manuscripts submitted to journals, and a subsequent need for more reviewers, leads to an increasing difficulty in finding suitable reviewers. Additionally, there is a perceived bias against women in the review process8.There are also cases of review manipulation, sometimes involving identity fraud which forced publishers to retract articles9. Overall, the review process is considered opaque and expensive, slowing down the speed of scientific discovery and progress."

Peer-review is the traditional and necessary process at the heart of quality control in science [9, 12], determining the destinies of articles’ publications and therefore of scientific dissemination. However, the current peer-review system is outdated: in the past, it was effective when scholarly communication happened exclusively through printed paper journals, but nowadays, with the high and fast-paced levels of articles productions, its slow and multistage process doesn’t keep up with the times.
\subsubsection{It's a slow multi-stage process.} Indeed, articles submitted to journals can take from months to years after editors’ first scrutiny, going back and forth several review rounds before acceptance for publications.
\paragraph{Traditional peer-review process} This section takes inspiration from the editorial processes at \emph{Nature Methods} and \emph{Wiley}.
\begin{enumerate}
    \item The author submits a manuscript to the journal.
    \item  An editorial team  assesses the paper for what concerns the requirements of the journal's Author's Guideline (composition, sections...).
    \item An Editor in Chief (EIC) is assigned to the manuscript and checks if the paper meets the scopes of the journal and novelty criteria. If the paper is rejected, it does not go through the consideration process and the editor notifies the author.
    \item If the editorial decision is to send the manuscript for review, the handling editor looks for potential reviewers. Potential reviewers can be suggested by the authors themselves, but usually it's up to the editor to find researchers in the field who are competent enough but not closely related to the study to review.
    \item Candidate reviewers receive the journal's invitation and evaluate acceptance based on their own level of expertise, intrinsic conflicts of interest and time availability. They then accept or decline. If they decline, they might suggest alternative reviewers. Editors keep searching for reviewers until they find two or three reviewers who agree to review the manuscript.
    \item When reviewers engage themselves in delivering peer-review report by a certain deadline, the author(s) are notified. The reviewers identity is usually only know to the editor but is hidden to the authors or the other reviewers themselves.
    \item Reviewers independently conduct their reviews by exposing in their reports strengths and weaknesses of the manuscript in a point-by-point fashion and submitting a final recommendation to the editor: acceptance, rejection or revision (major or minor).
    \item the handling editor or the editorial team considers all the submitted reports and makes the final decision.
    \item If the decision is a major or minor revision, authors are invited to re-submit a corrected or improved version of the manuscript with a letter to the reviewers. The same reviewers might be contacted again and if they want they can follow up the peer-review process.
    \item If the decision is a rejection, the editor might indicate another journal where to submit the manuscript.

\end{enumerate}
\subsubsection{It's unaccredited and unpaid.} This is mainly due to the fact that the reviewers that are normally appointed by authors and/or editors are actually full-time researchers themselves, who work on a volunteer basis taking time from their primary research. Though things are changing, and now more and more journals encourage a transparent peer review process with the publication of reviewers names and reports, in most of the cases, to guarantee an unbiased output, journals don’t even get credit researchers for this form of unpaid work. On top of this, author-level metrics that measure the scientific impact and productivity of academics, such as the h-index, and are taken into account by the funding agencies, are purely based on the number of citations per each publications while neglecting the full spectrum of scientific contributions (software, data collection, presentations, reviews…). Therefore, peer-reviewing is an intellectual investment without any external return for researchers’ career. A major consequence of not promoting incentives for the quality (and quantity) of peer-reviews is to either have good research unpublished (because unreviewed) and abandoned in preprint archives or bad science published through sloppy and uncritical reviews [5].
\subsubsection{It's vulnerable to bias and fraud.} Last but not least, the ‘publish or perish’ culture has been more and more inducing malicious behaviour during the peer-review process, such as attempts of scientific fraud (authors trying to review their own papers) and abuse (reviewers producing extremely harsh reviews to damage competitors by blocking the publication of their ideas). Finally, the current peer-review system is usually not double-blind, thus making its decisional process vulnerable to all forms of biases (gender bias, cultural bias, professional bias, etc...). These trust problems are one of the major issues facing scholarly communication.

\subsubsection{Types of peer-review}
\begin{itemize}
    \item single-blind peer review
    \item double-blind peer review
    \item transparent peer review
    open identities open reports open participation open interaction open platform ('decoupled review')
\end{itemize}



\subsection{Related work}
Credit for peer review is not a new thing: mostly journals handled it so far with virtual 'badges', certificates of performance, citation in annual editorials but none of these is standardized in a way that has a weight for funders world ranking bodies institutions like impact factor h-index.
In relation to security issues \emph{CryptSubmit} \emph{MaRSChain}
\emph{Crossref} primary DOI provider, announced that review reports could be now registered as part of their services
\emph{Publons} provides a way to keep track of the reviews records and integrate them into academic profiles such as ORCID
ScienceOpen
\emph{DEIP}
Rubriq provides independent and standardized peer review that can be accepted and shared across all types of journals and distribution systems. Now absorbed by Research Square.
\emph{Peerage of Science} that provides a journal-independent free service for scientific peer review and publishing. The interesting feature is that it provides also a peer-review of the reviews.
\emph{Blockchain for Peer Review,}pilot project sponsored by Digital Science and Katalysis, aims at developing a protocol for storing peer reviews submitted by publishers on a blockchain such that review process can be independently verified and review info is fed to ORCID researcher profiles []
\emph{ReviewCoin}
\emph{I8X Platform}
A more comprehensive reform of academic endorsement has been proposed in the manifesto ‘Towards Open Science:The Case for a Decentralized Autonomous Endorsement System11’, published anonymously.The author(s) propose a new academic endorsement system that is not based on current journal publication practices which are argued to be expensive, slow, disregard non-traditional output and negative results, and which give too much power to editors and publishers. Built on the blockchain, the Academic Endorsement System (AES) is based on a new form of currency, coined academic endorsement points (AEP), which can be used by scientists to reward scientific work that is worthy of endorsement. Moreover, the amount of AEP credited to a scientist is based on the AEP received for previous work. Researchers whose output has been endorsed to a high degree will have a larger influence in the community. Any kind of research output could be endorsed, for example blog posts, data sets, software etc.Another advantage is that this system would be faster than the time it takes for citation metrics to accrue.
The startup Matryx.ai13 aims to transform the nature of research collaboration itself. Its initiators claim that one of the problems with current scientific practices is its incentive structure. Scientific awards, for example, are based on competition rather than on collaboration, despite the fact that new ideas are usually collaborative in nature. Moreover, different researchers and research groups are often working on the same problems, wasting, as they claim on their website,‘brain power’, time and money. Matryx intends to provide a platform enabling and incentivising research collaboration while discouraging solitary and siloed research. On Matryx, a research project starts when users set a bounty on a problem that they want the community to solve. Contributors solve these problems by working with each other on the platform.The collaboration is tracked through the blockchain thus ensuring contributors are credited appropriately when the problem is solved. In principle, everyone can contribute in the collaboration: scientists, PhD students and even enthusiastic amateurs.


\subsection{Ideas here}

\section{System concept}
In this paper we propose Ants-Review, a new incentivisation mechanism built on a distributed platform that issues open peer-reviews to validate scientific papers while preserving the anonymity of its contributors. We imagine a final paper originating from the peer-review process as a complex system that emerges from the interactions between the authors and the reviewers, a whole that is more than the sum of its parts. Therefore, the name evokes an ant colony as a self-organising organism in which all micro-contributions of the individuals emerge into complex behaviour.
A previous version of this paper can be found here \cite{AntsReview}, as a proposal submitted for the ETH Turin 2020 Hackathon.
and its design and attempt of implementation are exposed in the following section.
Here we expose the main traits/features of AR
\begin{itemize}
\item accountability: both for recognition and for security vs fraud abuse etc we need to be able to track back the identities of the contributors to a peer-review to make sure there are no conflict of interest. this means the platform acts like a version control system where commit are permanent and hashed 
\item transparency of the records: all reviews should be open access to the community to prevent waste of knowledge and disrupt the secretive system of black boxes of peer reviews in journals that centralize all the reports 
\item incentivisation: via symbolic and material recognition of the performed work. reputation and bounties
\item inclusiveness via gamification: all the community is involved in the process - no selection process feeling of belonging -- quadratic voting and share of the stake
\end{itemize}
crowdfunding schemes leveraged by blockchain technology
In the following sections we describe the architecture of the system and its building blocks
AntsReview [Fig.\ref{fig:contracts}] is a distributed peer-review system which synthesizes successful ideas from previous systems including The Bounty Network \cite{Bounty}, ERC-20 \cite{ERC20}, AZTEC Protocol \cite{AZTEC}, IPFS \cite{IPFS}, Proof of Existence \cite{TimeStamp-Haber}, ...

\subsection{Design}

There are two possible scenarios depending if we save the role of the editor and we integrate an existing journal or if AntsReview becomes a platform completely decoupled from journals.

\subsubsection{AnstReview process: scenario 1 (journal-dependent)}

\begin{enumerate}
    \item The author submits a manuscript to the journal.
    \item If the papers passes stage 2-3 of editorial decision (appropriateness of style and coherence with the journal's scope)
    \item the editor issues a bounty (\emph{bounty issuance}) for the paper review. Fulfillment rules are set in a smart-contract.
    \item (footnote?) the bounty prize could correspond to part of the article submission fees and article publication fees requested to the authors by the journal.
    \item Reviews fulfilling the smart-contract requirements are validated by the journal editors.
    \item When the reviews are accepted (\emph{bounty fulfillment}), reviewers awarded a token called \emph{ANT}.
    \item Not all reviews will have the same quality and depth. However, each of them might bring up one or more relevant contribution for the sake of improving the manuscript. In order to redistribute the bounty in a fair manner across reviewers, reviews will be voted by Anters (members of the community) via Quadratic Voting to rank the most prolific peer-reviewers.

\end{enumerate}

\subsubsection{AnstReview process: scenario 2 (community-driven)}:

\begin{enumerate}
    \item The author submits a manuscript to the pre-print service or to an hypothetical AntsReview platform).
    \item The paper is accepted by the community for revision in its whatever format.
    \item The author issues a bounty (\emph{bounty issuance}) for the paper review. Fulfillment rules are set in a smart-contract.
    \item (footnote?) the bounty prize could come by a community stake that participates in the peer-review process either as a reviewer or as a reviewer checker.
    \item Reviews fulfilling the smart-contract requirements are validated by the AntsReview community/contributors.
    \item When the reviews are accepted (\emph{bounty fulfillment}), reviewers awarded a token called \emph{ANT}.
    \item Not all reviews will have the same quality and depth. However, each of them might bring up one or more relevant contribution for the sake of improving the manuscript. In order to redistribute the bounty in a fair manner across reviewers, reviews will be voted by the community via Quadratic Voting.
    \item In order to incentive 'fair play' also the review checkers who vote for the best quality reviews will share the final prize from the bounty. Since it's in their own interest to 'bet' on a good review, this gamed system will discourage internal dynamics prompt to favour specific members of the community beyond real personal merit. It will also fuel ethical behaviour.

 \end{enumerate}

\subsection{Implementation}

\begin{figure}
\centering
\includegraphics[scale=0.28]{AntsReview}
\caption{AntsReview Smart Contracts}
\label{fig:contracts}
\end{figure}
In this section we provide an overview of how we implemented the system by focusing on its components.

The AntsReview Protocol is divided into different modules responsible for the following functionalities as shown in the flow-chart [Fig. \ref{fig:contracts}]:

\begin{itemize}
\item AntsReview - manage access management and the basic system.
\item Tokenomics - manage system incentivization mechanism integrating ideas from DeFi and Quadratic Funding.
\item Privacy - maintains the anonymity of the system via AZTEC Protocol.
\end{itemize}

\subsubsection{AntsReview}

AntsReview [\ref{appendix:b}] is the core of the smart contracts deployed on Ethereum Rinkeby's Testnet \cite{Rinkeby}
\newline It implements a Bounty-like system where an issuer (author) is asked a series of actions, to ensure transparency and robustness, regarding the AntReview that he's creating:
\begin{itemize}
  \item to upload the file containing the requirements of the peer-review and the paper into IPFS\cite{IPFS}.
  \item to specify a deadline in the form of a unix timestamp after which the fulfillment will no longer be accepted.
  \item to send the amount of ether for the reward.
\end{itemize}

AntsReviewRoles[\ref{appendix:a}], to ensure integrity of the system, implements an access management, leveraging on AccessControl.sol by OpenZeppelin Library \cite{OZ}, through the creation of the Roles Issuer and Peer-Reviewer.
\newline It also integrates a circuit breaker design pattern via Pausable.sol by OpenZeppelin to allow the Pauser Role, that by default is the Owner of the Smart Contracts, to pause (or unpause) the functions in case of emergency.
\newline In this way, only the Issuer (author) can create a new AntReview.
\newline Once created, an AntReview is available to be fulfilled by Peer-Reviewers, uploading the peer-review on IPFS before the deadline.
\newline The AntReview after being validated is then accepted by the Issuer and paid equally between Peer-Reviewers.
\newline Also, the Issuer can cancel the AntReview at any time and withdraw the amount of ether staked.
\newline One of the key aspects of the protocol is the concept of Proof of Existence, introduced by Stuart Haber \& W. Scott Stornetta with the paper "How to time-stamp a digital document"\cite{TimeStamp-Haber}
\newline As the pioneers of blockchain an cited by Bitcoin whitepaper, they were in a mission to solve the problem of immutability of digital records that became reality with the Blockchain and Bitcoin\cite{Bitcoin} as its first application.
\newline In Ethereum \cite{Ethereum} the concept of immutability is achieved via the blockchain and proof of work consesus algorithm by a chain of timestamped block hashes that are secured by miners competing computations.
\newline AntsReview is evaluating the advantages of storing the hashes of the peer-reviews into a Merkle Tree \cite{BayerHaber1992} to achieve immutability and efficiency in verifying the data with a complexity of O(logn) for searching through the tree.
\newline A Merkle tree is a particular type of binary tree, formed by a set of nodes with a large set of leaf nodes at the bottom of the tree containing representing the hash of the original data, a set of intermediate nodes where each node is the sum of the two children hashes, ending with a single root hash.
\newline In order to verify if a specific hash is included into a Merkle Tree, leveraging on MerkleProof.sol by OpenZeppelin Library, we need to provide a proof, generated from the leaf of interest, that represent the remification derived from the leaf, the root hash and finally the leaf we want to verify.
\newline The paper and peer-review documents will be included into a Merkle Tree by hashing the data using keccak256, a cryptographic hashing algorithm available in Solidity \cite{Solidity}, an object-oriented programming language for writing smart contracts that run on the Ethereum Virtual Machine (EVM).

\subsubsection{Tokenomics}

AntsReview has few tokens, each of which plays an integral role in the functioning and anonymity of the decentralized protocol.
\newline \textbf{Ant}, is the primary protocol token and can be staked into an AntReview if you wish to provide the protocol with an additional security promise.
\newline Ant [\ref{appendix:c}] is implemented by inheriting ERC20PresetMinterPauser.sol from OpenZeppelin Library with name: \textbf{\emph{Ant}} ans symbol: \textbf{\emph{ANT}}.
\newline An AntReview, issued by an author will be seen as a poll where Anters (members of the AntsReview Community) can stake ether or tokens from DeFi services including Dai \cite{Dai}, Chai \cite{Chai}, ... with the possibility to accrue interest for the duration of an AntReview (1 year on average), via MakerDAO DSR (Dai Saving Rate) \cite{DSR} or Compound \cite{Compound} to cite a few.
\newline An \textbf{AntsReview DAO} will be formed in the future to allow Ant stakers the ability to participates in a Decentralized Autonomous Organization which will be used to govern important aspects of the decentralized protocol, from smart contracts upgrades to more minor changes in settings across the protocol.
\newline \textbf{Hive}. When an Anter deposits into an AntReview, they will instantly receive the Hive token which represents their deposit and the accrued interest it gains over time in the AntReview.
\newline This token doesn't need to be locked in the network and it can be traded, sold, or held as the Anter desires.
\newline Hive will be implemented using a similar approach to Ant and Chai, a token that accrues interest from MakerDAO's DSR or Compound, with name: \textbf{\emph{Hive}} and symbol: \textbf{\emph{HIVE}}
\newline Ideally, but is still under investigation, after the deadline of the AntReview is reached, the Hive token will be burned for Ant and wrapped into zkAnt to promote anonymity and sent equally to the peer-reviewers that fulfilled the AntReview.
\newline \textbf{zkAnt}, is an Ant token wrapped into AZTEC Protocol \cite{AZTEC} ZK-SNARKs, that will be extended to PLONK \cite{PLONK} Proof to collapse the gas costs of private transactions on mainnet.
\newline zkAnt will be implemented via AZTEC 2.0 (still in development) that will allow to securely wrap the Ant Token into zkAnt, allowing private transactions on mainnet and preserving the anonymity of Peer Reviewers.
\newline Finally, another concept under investigation is Quadratic Funding \cite{LiberalRadicalism}, used by Gitcoin \cite{Gitcoin} to match donations for Grants. This will allow to incentives prolific Peer-Reviewers and improve the stability of the system together with ideally a reward system to rank best Peer-Reviewers.

\subsubsection{Privacy}

The Anonymity of an agent in the system is achieved in two ways:

\begin{itemize}
  \item PseudoAnonimity granted by an Ethereum's Externally Owned Accounts (EOA) address that can pseudo-obscure the identity of the agent.
  \item Private Transactions allowed by AZTEC Protocol security layer via zkAnt. Future developments will allow to leverage on PLONK Proofs to reduce gas costs and improve scalability.
\end{itemize}

 Externally Owned Accounts (EOA) in Ethereum are controlled by private keys, are 20 bytes long and have no code. An user can send messages from an externally owned account by creating and signing a transaction.
\newline The limitation of such solution regarding the Privacy of the Protocol is due to the fact that blockchain and transactions are public so anyone can see the details of a transaction, browsing a block explorer like Etherscan, and also they're subjected to data mining to extract value and identify users in the blockchain.
\newline To avoid that, AntsReview will implement zkAnt, an Ant token wrapped into AZTEC Protocol, where leveraging on zkSNARKS, and PLONK in the future, will allow private transactions in the system to maintain the privacy and anonimity of Peer-Reviewers.
\newline The AZTEC Protocol was conceived to enable privacy on public blockchains. It uses ZK-SNARKS to validate encrypted transactions.
\newline ZK-SNARKs (Zero-Knowledge Succinct Non-Interactive Argument of Knowledge) \cite{ZKSNARKs} refers to a proof construction where one can prove possession of certain information, like a secret key, without revealing that information, and without any interaction between prover and verifier.
\newline Zero-Knowledge \cite{ZeroKnoledge} proofs allow an actor (prover) to prove to another (verifier) that a statement is true, without revealing any information apart from the validity of the statement itself.
\newline Future developments will investigate viable solutions, for scaling the system, like ZK-Rollups by AZTEC Protocol, one of the options being developed for a type of layer 2 construction that runs on top of Ethereum to improve scalability. This will enable to run smart contracts at scale with quasi-instant transactions while still being secured by Ethereum.
\subsubsection{Privacy/anonymity}. To enhance the peer-reviewer's privacy, we would like to implement zkANT, an ERC20 token, Ant, wrapped into AZTEC Protocol ZK-SNARKs, a non-interactive type of zero-knowledge privacy protocol to enable private transactions on Ethereum. Zero-knowledge proof is a mathematical cryptographic method that through values permutations allows one party (the prover) to prove to another (the verifier) the veracity of a statement, without having to reveal what the statement is. We will use ZK-SNARKs (which stands for “Zero-Knowledge Succinct Non-Interactive Argument of Knowledge”) via the open source libraries of AZTEC protocol. AZTEC, now TURBOPLONK bla bla (faster?) combines the confidentiality of the transactions (the value of the transaction is hidden, the amount of token (?) is not revealed thus preventing biases in the peer-review quadratic voting process = "I want to bet on this review because I want to win the bounty") with general-purpose computation (it's a Turing-complete Virtual Machine): this means transaction are not limited just to the local cryptocurrency (in our case ANT) but cross-asset trades are allowed (I can exchange ANT tokesn with other crypto-currencies).
\section{Discussions}
\subsection{A new community-driven standard?}
- a change is already happening: specially new born publishers are opening up review process (BioMed Central, ELife, Frontiers, PeerJ, F1000 Research) -- from 2020 also Nature declared that will do the same.

- new business model? at some point pre-print platforms (dissemination) might want to integrate peer-review (validation) into their platforms - dissociation of initial scientific dissemination and scientific validation will force publishing industry to adapt: how will they justify their added value? as Tennant points out: branding archiving

- peer reviews requires a standard: building smart-contracts for peer-reviews might set a first step in this direction

- community-organized peer reviews: peer reviews will be more and more out of the hands of publishers, researchers will be the ones seeking the papers to review (like in Rubriq).

future of quality control and moderation: open participation model, community-driven | certification and reputation: expansion of reviewer pool with fewer barriers to enter and more inclusion of young scientists, reputation build within community and not journals

- portable peer review not only within a consortium of journals

-value of peer review as 'academic capital' recognized by research funders and hiring committees

-a bit off topic but: automated peer review (services such as StatReviewers) might fasten the formal aspects of the job allowing reviewers to focus on the content.

\subsubsection{Perspectives for a Decentralized Autonomours Organization (DAO)}

\section{Conclusion}
In this paper we addressed a crucial problem within scholarly academic communication: the peer-review process. We showed how blockchain technology could provide an efficient and viable solution and open up possible directions for a change in paradigm in scientific communication. We proposed an incentive mechanism that could solve the problems of lack of acknowledgment and trust during peer-review. We exposed the architecture of our project and our Proof of Concept (PoC) for which we adopted cutting-edge tools from the open source blockchain community.

\subsubsection{Supplementary material}
TBD

\subsubsection{CRediT Authorship Contribution Statement}
TDB

\subsubsection{Declaration of Competing Interest}
The authors declare no competing interests.

\subsubsection{Acknowledgements} We would like to thank Matteo A. Tambussi and Keno Budde for internal review and useful comments.

\newpage

\section{Appendix}

\appendix

\section{AntsReviewRoles.sol}

\label{appendix:a}

\begin{lstlisting}[language=Solidity]

  /// SPDX-License-Identifier: GPL-3.0
  pragma solidity 0.6.8;

  ///@title AntsReview
  ///@author Nazzareno Massari
  ///@notice AntsReviewRoles Access Management for Issuer and Peer-Reviewer
  ///@dev All function calls are currently implemented without side effecs through TDD approach
  ///@dev OpenZeppelin library is used for secure contract development

  import "@openzeppelin/contracts/access/Ownable.sol";
  import "@openzeppelin/contracts/access/AccessControl.sol";
  import "@openzeppelin/contracts/utils/Pausable.sol";

  contract AntsReviewRoles is Ownable, AccessControl, Pausable {

    /// Roles
    bytes32 public constant ISSUER_ROLE = keccak256("ISSUER_ROLE");
    bytes32 public constant PEER_REVIEWER_ROLE = keccak256("PEER_REVIEWER_ROLE");
    bytes32 public constant PAUSER_ROLE = keccak256("PAUSER_ROLE");


    constructor() public {
            _setupRole(DEFAULT_ADMIN_ROLE, owner());
            _setupRole(PAUSER_ROLE, owner());
    }

      /// Modifiers
      modifier onlyAdmin() {
          require(isAdmin(msg.sender), "Caller is not an admin");
          _;
        }

      modifier onlyIssuer() {
          require(isIssuer(msg.sender), "Caller is not an issuer");
          _;
      }

      modifier onlyPeerReviewer() {
          require(isPeerReviewer(msg.sender), "Caller is not a peer-reviewer");
          _;
      }

      /// Functions

      function isAdmin(address account) public view returns (bool) {
        return hasRole(DEFAULT_ADMIN_ROLE, account);
      }

      function isIssuer(address account ) public view returns (bool) {
        return hasRole(ISSUER_ROLE, account);
      }

      function isPeerReviewer(address account ) public view returns (bool) {
        return hasRole(PEER_REVIEWER_ROLE, account);
      }

      function addIssuer(address account) public onlyAdmin returns (bool) {
        require(!isIssuer(account), "Account is already an issuer");
        grantRole(ISSUER_ROLE, account);
        return true;
      }

      function addPeerReviewer(address account) public onlyAdmin returns (bool) {
        require(!isPeerReviewer(account), "Account is already a peer-reviewer");
        grantRole(PEER_REVIEWER_ROLE, account);
        return true;
      }

      function removeIssuer(address account) public onlyAdmin returns (bool) {
        require(isIssuer(account), "Account is not an issuer");
        revokeRole(ISSUER_ROLE, account);
        return true;
      }

      function removePeerReviewer(address account) public onlyAdmin returns (bool) {
        require(isPeerReviewer(account), "Account is not a peer-reviewer");
        revokeRole(PEER_REVIEWER_ROLE, account);
        return true;
      }

      /// @notice Pause all the functions
      /// @dev the caller must have the 'PAUSER_ROLE'
      function pause() public {
        require(hasRole(PAUSER_ROLE, msg.sender), "BadgeFactory: must have pauser role to pause");
        _pause();
      }

      /// @notice Unpause all the functions
      /// @dev the caller must have the 'PAUSER_ROLE'
      function unpause() public {
            require(hasRole(PAUSER_ROLE, msg.sender), "BadgeFactory: must have pauser role to unpause");
            _unpause();
      }
  }

\end{lstlisting}

\newpage
\section{AntsReview.sol}

\label{appendix:b}

\begin{lstlisting}[language=Solidity]
  /// SPDX-License-Identifier: GPL-3.0
  pragma solidity 0.6.8;

  ///@title AntsReview
  ///@author Nazzareno Massari
  ///@notice AntsReview to allows issuer to issue an antReview which peer-reviewers can fulfill
  ///@dev All function calls are currently implemented without side effecs through TDD approach
  ///@dev OpenZeppelin library is used for secure contract development

  import "./AntsReviewRoles.sol";
  import "@openzeppelin/contracts/math/SafeMath.sol";
  import "@openzeppelin/contracts/utils/Address.sol";
  import "@openzeppelin/contracts/utils/Counters.sol";

  interface AntToken {
    function transfer(address reciptient, uint amount) external returns (bool);
    function balanceOf(address account) external view returns (uint);
  }

  contract AntsReview is AntsReviewRoles {

    using SafeMath for uint256;
    using Address for address payable;
    using Counters for Counters.Counter;

    /// Enums
    enum AntReviewStatus { CREATED, ACCEPTED, CANCELLED }

    /// Token
    AntToken internal ant;

    /// Counter
    Counters.Counter private _antReviewIdTracker;

    /// Storage
    AntReview[] public antreviews;

    mapping(uint256 => Peer_Review[]) peer_reviews;

    /// Structs
    struct AntReview {
        address payable issuer;
        uint256 deadline;
        string ipfs_hash;
        AntReviewStatus status;
        uint256 amount; //in wei
    }

    struct Peer_Review {
        bool accepted;
        address payable peer_reviewer;
        string ipfs_hash;
    }


    /// Events

    event AntReviewIssued(address issuer, uint256 amount, string ipfsHash);
    event AntReviewFulfilled(uint256 antReviewId, address peer_reviewer, uint256 peerReviewId, string ipfsHash);
    event AntReviewAccepted(uint256 antReviewId, address issuer, address peer_reviewer, uint256 indexed peerReviewId, uint256 amount);
    event AntReviewCancelled(uint256 indexed antReviewId, address indexed issuer, uint256 amount);

    constructor(address ant_) public {
      ant = AntToken(ant_);
    }

    /// Fallback

    fallback() external payable {
      revert();
    }

    receive() external payable {
      revert();
    }

    /// Modifiers

    modifier hasValue() {
        require(msg.value > 0);
        _;
    }

    modifier antReviewExists(uint256 _antReviewId){
      require(_antReviewId < antreviews.length);
      _;
    }

    modifier peerReviewExists(uint256 _antReviewId, uint256 _peerReviewId){
      require(_peerReviewId < peer_reviews[_antReviewId].length);
      _;
    }

    modifier hasStatus(uint256 _antReviewId, AntReviewStatus _desiredStatus) {
      require(antreviews[_antReviewId].status == _desiredStatus);
      _;
    }

    modifier peerReviewNotYetAccepted(uint256 _antReviewId, uint256 _peerReviewId) {
      require(peer_reviews[_antReviewId][_peerReviewId].accepted == false);
      _;
    }

    modifier validateDeadline(uint256 _newDeadline) {
        require(_newDeadline > now);
        _;
    }

    modifier isBeforeDeadline(uint256 _antReviewId) {
      require(now < antreviews[_antReviewId].deadline);
      _;
    }


    ///@notice Instantiates a new AntReview
    ///@dev Access restricted to Issuer
    ///@param _deadline The unix timestamp after which fulfillments will no longer be accepted
    ///@param _ipfsHash The IPFS Hash of the Scientific Paper
    ///@return True If the antReview is successfully issued
    function issueAntReview(
        string calldata _ipfsHash,
        uint64 _deadline
    )
        external
        payable
        hasValue()
        validateDeadline(_deadline)
        onlyIssuer()
        whenNotPaused()
        returns (bool)
    {
        _antReviewIdTracker.increment();
        antreviews.push(AntReview(msg.sender, _deadline, _ipfsHash, AntReviewStatus.CREATED, msg.value));
        emit AntReviewIssued(msg.sender, msg.value, _ipfsHash);
        return true;
    }


    ///@notice Submit a fulfillment for the given antReview
    ///@dev Access restricted to Peer-Reviewer
    ///@param _antReviewId The index of the antReview to be fufilled
    ///@param _ipfsHash The IPFS Hash which contains evidence of the fufillment
    ///@return True If the AntReview is successfully fulfilled
    function fulfillAntReview(uint256 _antReviewId, string memory _ipfsHash)
      public
      antReviewExists(_antReviewId)
      onlyPeerReviewer()
      hasStatus(_antReviewId, AntReviewStatus.CREATED)
      isBeforeDeadline(_antReviewId)
      whenNotPaused()
      returns (bool)
    {
      peer_reviews[_antReviewId].push(Peer_Review(false, msg.sender, _ipfsHash));
      emit AntReviewFulfilled(_antReviewId, msg.sender, (peer_reviews[_antReviewId].length.sub(1)),_ipfsHash);
      return true;
    }


    ///@notice Accept a given Peer-Review
    ///@dev Access restricted to Issuer
    ///@param _antReviewId the index of the antReview
    ///@param _peerReviewId the index of the fulfillment being accepted
    ///@return True If the AntReview is successfully being accepted
    function acceptAntReview(uint256 _antReviewId, uint256 _peerReviewId)
        public
        antReviewExists(_antReviewId)
        peerReviewExists(_antReviewId,_peerReviewId)
        onlyIssuer()
        hasStatus(_antReviewId, AntReviewStatus.CREATED)
        peerReviewNotYetAccepted(_antReviewId, _peerReviewId)
        whenNotPaused()
        returns (bool)
    {
        peer_reviews[_antReviewId][_peerReviewId].accepted = true;
        antreviews[_antReviewId].status = AntReviewStatus.ACCEPTED;
        peer_reviews[_antReviewId][_peerReviewId].peer_reviewer.sendValue(antreviews[_antReviewId].amount);
        emit AntReviewAccepted(
          _antReviewId,
          antreviews[_antReviewId].issuer,
          peer_reviews[_antReviewId][_peerReviewId].peer_reviewer,
          _peerReviewId, antreviews[_antReviewId].amount
        );
        return true;
    }


    ///@notice Cancels the antReview and send the funds back to the issuer
    ///@dev Access restricted to Issuer
    ///@param _antReviewId the index of the antReview
    ///@return True If the AntReview is successfully cancelled
    function cancelAntReview(uint256 _antReviewId)
        public
        antReviewExists(_antReviewId)
        onlyIssuer()
        hasStatus(_antReviewId, AntReviewStatus.CREATED)
        whenNotPaused()
        returns (bool)
    {
        antreviews[_antReviewId].status = AntReviewStatus.CANCELLED;
        antreviews[_antReviewId].issuer.sendValue(antreviews[_antReviewId].amount);
        emit AntReviewCancelled(_antReviewId, msg.sender, antreviews[_antReviewId].amount);
        return true;
    }

  }

\end{lstlisting}

\newpage
\section{Ant.sol}
\label{appendix:c}

\begin{lstlisting}[language=Solidity]
  /// SPDX-License-Identifier: GPL-3.0
  pragma solidity 0.6.8;

  ///@title AntsReview
  ///@author Nazzareno Massari
  ///@notice Ant ERC20 Token
  ///@dev All function calls are currently implemented without side effecs through TDD approach
  ///@dev OpenZeppelin library is used for secure contract development

  import "@openzeppelin/contracts/presets/ERC20PresetMinterPauser.sol";


  contract Ant is ERC20PresetMinterPauser {

    constructor()
    ERC20PresetMinterPauser("Ant", "ANT")
    public {

    }

  }
\end{lstlisting}

\newpage

% ---- Bibliography ----
%
% BibTeX users should specify bibliography style 'splncs04'.
% References will then be sorted and formatted in the correct style.
%
\bibliographystyle{splncs04}
\bibliography{ref1-intro.bib, ref2-peerreview.bib, ref2-relatedwork.bib, ref3-system.bib}

\end{document}
